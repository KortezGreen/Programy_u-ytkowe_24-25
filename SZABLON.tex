\documentclass[a4paper,11pt]{article}
\title{Dokument}
\author{Kacper Kieplin}
\date{ \today}
\usepackage[MeX]{polski}
\usepackage[utf8]{inputenc}
\usepackage{hyperref}
\usepackage{graphicx}
\usepackage{amsmath}
\newtheorem{twr}{Twierdzenie}
\newtheorem{lem}[twr]{Lemat}

\usepackage[a4paper, left=3.5cm, right=3.5cm, top=3.5cm, bottom=3.5cm, headsep=1.2cm]{geometry} 

\begin{document}
\maketitle
\begin{center}
\tableofcontents
\end{center}

\section{rozdział}
\subsection{podrozdział}
\subsubsection{podpodrozdział} 

Wewnątrz akapitu suma może być napisana jako $\sum^n_{k=1}a_n$ albo jako $\displaystyle\sum^n_{k=1}a_n$, iloczyn jako $\prod^n _{k=2}$ albo $\displaystyle\prod^n _{k=2}$

tekst\\
\url{http://www.rich.edu/˜bush}\\
Jest $-30\,^{\circ}\mathrm{C}$.\\
\emph{kursywa}\\
zielony\footnote{niebieski i żółty}\\

\label{etykieta}
test
\ref{etykieta}
test
\pageref{etykieta}\\

\begin{displaymath}
a^x+y \neq a^{x+y}
\end{displaymath}\\

\begin{equation}
a^x+y \neq a^{x+y}
\end{equation}\\

\begin{equation}
\epsilon > 0 \label{eq:eps}
\end{equation}
Ze wzoru (\ref{eq:eps})
otrzymujemy \ldots\\

\begin{displaymath}
\lim_{n \to \infty}
\sum_{k=1}^n \frac{1}{k^2}
= \frac{\pi^2}{6}
\end{displaymath}\\


\begin{lem} Pierwszy
lemat\dots\label{lem:1} \end{lem}
\begin{twr}[Dyzma]
Przyjmując w~lemacie~\ref{lem:1},
że $\epsilon=0$\dots \end{twr}
\begin{lem}Trzeci lemat\end{lem}

\begin{figure}
\centering
\includegraphics[angle=360,width=0.5\textwidth]{Kwiaty.jpg}
\end{figure}

\end{document}
