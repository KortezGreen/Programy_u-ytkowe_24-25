\documentclass[a4paper,11pt]{amsart}
\title{PRACA DOMOWA $N\,^{\circ}\mathrm{}$ 1}
\author{Kacper Kieplin}
\usepackage[MeX]{polski}
\usepackage[utf8]{inputenc}
\usepackage{setspace}
\usepackage{hyperref}
\usepackage[a4paper, left=3.5cm, right=3.5cm, top=3.5cm, bottom=3.5cm, headsep=1.2cm]{geometry} 
\singlespacing 
\begin{document}
\maketitle
\begin{spacing}{1.4}

\noindent \section{Przykład tekstu po-angielsku: Aesop ,,The Hare and the Tortoise''}
\end{spacing}
\begin{spacing}{1.0}
One day the Hare laughed at the short feet and slow speed of the Tortoise. The
Tortoise replied:

\textit{,,You may be as fast as the wind, but I will beat you in a race''}

The Hare thought this idea was impossible and he agreed to the proposal. It\\
was agreed that the Fox should choose the course and decide the end.

The day for the race came, and the Tortoise and Hare started \textbf{together}.

\textbf{The Tortoise never stopped for a moment}, walking slowly but steadily,
right to the end of the course. The Hare ran fast and stopped to lie down for a rest.
But he fell fast asleep. Eventually, he woke up and ran as fast as he could. But
when he reached the end, he saw the Tortoise there already, sleeping comfortably
after her effort.\\
\end{spacing}

\section{Rozmiary czcionki}


My przeczytaliśmy wszystkie polecenia na stronie:

\url{https://www.overleaf.com/learn/latex/Font_sizes%2C_families%2C_and_styles.}

Teraz możemy to wykorzystać.

{\tiny Czcionka jest tutaj bardo mała, trzeba powiększyć ten tekst.} {\small Teraz już lepiej, ale jeszcze nie\\
zwykle.} O, tutaj jest normalny tekst! {\large A tutaj jest większa czcionka dla czegoś\\
specjalnego.} {\huge I, oczywiście, teraz spróbujemy my\\
największą czcionką.}



\end{document}
